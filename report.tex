\documentclass[]{article}

\usepackage[margin=1.0in]{geometry}

% Line numbers
\usepackage{lineno}
\linenumbers

% Figures
\usepackage{graphicx}
\graphicspath{{figures/}}

% Trick Latex into placing all figures and tables at the end of the document
% regardless of their position in the source code.
\renewcommand{\textfraction}{1.0}
\renewcommand{\floatpagefraction}{0.9}

% Equations
\usepackage{amsmath}

% Bibliography
\usepackage{natbib}

% Better tables
\usepackage{booktabs}
\usepackage{multirow}
\newcommand{\ra}[1]{\renewcommand{\arraystretch}{#1}}


%--------------------------------------------------
% TITLE
%--------------------------------------------------
\title{SIO 217A: Cloud Droplet Growth}
\author{Team MecheE \\ Bengu Ozge Akyurek, David Larson, Guangchao Wang}
\date{\today}

\begin{document}

\maketitle


%--------------------------------------------------
% INTRO
%--------------------------------------------------
\section{Introduction}
Atmospheric thermodynamics focuses on water and its transformations. Advanced
topics are usually focuses on phase transitions of water, homogeneous and
inhomogeneous nucleation, effect of dissolved substances on cloud condensation,
role of supersaturation on formation of ice crystals and cloud droplets.

Among those sub-topics, cloud nucleation is one area that
many researchers have worked on in the past serveral decades. Nucleation is
continually occurring in the atmosphere, both during gas-to-particle production
processes and in cloud formation which is a heterogeneous nucleation process
and involves nucleation centers or cloud nuclei.

People want to understand this process whereby a stable element of a new phase
first appears within the initial or parent phase for several reasons, one of
which is that the clouds have a very high albedo, and thus tend to reflect
incoming shortwave sunlight back into space. Another reason that affects
atmospheric thermodynamics is that the nucleated clouds, as a new phase, could
produce precipitation, which is the primary route for water to move
geographically around the globe within the water cycle. To begin understanding
the larger effects of clouds, it is useful to analyze the growth of droplets in
warm clouds. This report discusses and evaluates one simplified model of
droplet growth rate.


%--------------------------------------------------
% MODEL
%--------------------------------------------------
\section{Cloud Droplet Growth}

\subsection{Model Description}
In order to model the cloud droplet growth in time, we have used
Equation~\eqref{eq:5.26} from \cite{Curry}. This equation is based on
\cite{Mason}, which is based on \cite{Best}. The approximation assumes that a
droplet has the shape of a sphere around a saturated nucleus core. Furthermore,
the environment conditions are assumed to be constant, and the solute and
curvature effects are neglected. The initial approximation is obtained by
applying diffusion equations to the droplet, as shown in
Equation~\eqref{eq:Diffusion}.

\begin{equation}
    \label{eq:Diffusion}
    \dfrac{dm}{dt}=4 \pi r D_{v} \left( \rho_{v}(\infty) - \rho_{v}(r) \right)
\end{equation}

Adding the latent heat by condensation, shown in
Equation~\eqref{eq:LatentHeat}, gives the basis of the droplet growth rate
model.

\begin{equation}
    \label{eq:LatentHeat}
    \dfrac{dQ}{dt}=-L_{lv}\dfrac{dm}{dt}=-4 \pi r \kappa \left( T(\infty) - T(r) \right)
\end{equation}

These equations are combined to approximate the growth rate of a droplet by
diffusion as
\begin{align}
    \label{eq:5.26}
    r \frac{dr}{dt} = (S - 1) \left[ \frac{L_{lv}^2 \rho_l}{\kappa R_v T^2} + \frac{\rho_l R_v T}{e_s(T) D_v} \right] ^{-1} = \frac{S - 1}{K + D}
\end{align}
where $K$ and $D$ are the thermodynamic terms associated with heat conduction
and diffusion of water vapor, respectively.

Assuming the atmospheric ambient conditions are constant (i.e. $S$, $K$, and
$D$ are constant), then Equation~\ref{eq:5.26} can be integrated to get
\begin{align}
    \label{eq:5.27}
    r(t) = \left[ r_0^2 + \frac{2(S -1)}{K + D}(t - t_0)^2 \right] ^{1/2},
\end{align}
which can then be rearranged to find $t$

\begin{align}
    \label{eq:5.27T}
    t = (r^2 - r_0^2) \frac{K + D}{2(S - 1)}
\end{align}

Using the values from Table~\ref{tab:parameters}, the main results in
Table~\ref{tab:Table5.5} can be approximated. But to include the effects of
curvature and the solute, we must modify our model.


\begin{table}[h]
    \centering
    \caption{Parameters for droplet growth rate.}
    \label{tab:parameters}

    \begin{tabular}{l l l l}
    \toprule
    Parameter & Value & Units & Notes\\
    \midrule
    $S - 1$   & 0.05            & \%                         & \\
    $p$       & 900             & $mb$                       & \\
    $T$       & 273             & $K$                        & \\
    $r_0$     & 0.75            & $\mu m$                    & \\
    $L_{lv}$  & 2.5x10$^{6}$    & $J \ kg^{-1}$              & pure water at $T=273 K$ \\
    $\rho_l$  & 1000            & $kg \ m^{-3}$              & pure water at $T=273 K$ \\
    $R_v$     & 461             & $J \ kg^{-1} K^{-1}$       & \\
    $\kappa$  & 2.4x10$^{-2}$   & $J \ m^{-1} s^{-1} K^{-1}$ & $T=273 K$ \\
    $D_v$     & 2.21x10$^{-5}$  & $m^2 s^{-1}$               & $T=273 K$, $p=1000 mb$ \\
              & 2.46x10$^{-5}$  & $m^2 s^{-1}$               & $T=273 K$, $p=900 mb$ \\
    $e_s (T)$ & 6.15            & $mb$                       & $T=273 K$\\
    \bottomrule
    \end{tabular}
\end{table}


\begin{table}[h]
    \centering
    \caption{Growth rate of droplets with nuclei of NaCl, ($S - 1$)=0.05\%,
        $p$=900mb, $T$=273K, and $r_0$=0.75$\mu$m recreated from Table 5.5 of \cite{Curry}.}
    \label{tab:Table5.5}

    \ra{1.2}
    \begin{tabular}{@{} c r r r @{}}
        \\
        \toprule
        m [g] & $10^{-14}$ & $10^{-13}$ & $10^{-12}$ \\
        \midrule
        r [$\mu$m] & \multicolumn{3}{c}{t [s]} \\
        \midrule
        1  & 2.4    & 0.15   & 0.013 \\
        2  & 130    & 7.0    & 0.61 \\
        4  & 1,000  & 320    & 62 \\
        10 & 2,700  & 1,800  & 870 \\
        20 & 8,500  & 7,400  & 5,900 \\
        30 & 17,500 & 16,000 & 14,500 \\
        50 & 44,500 & 43,500 & 41,500 \\
        \bottomrule
    \end{tabular}
\end{table}


One of the main properties of Equation~\eqref{eq:5.27T} is that it does not
depend on the nucleus mass. This led us to the first modification on the
approximation equation. Rather than starting from $r=r_{0}=0.75\ 10^{-6} \mu m$
and $t_{0}=0$, we have altered the definition of the starting radius as the
radius of the nucleus without any water. We haven't changed the value of
$r_{0}$, as this was given as one of the constants. But, we have defined
$t_{0}$ as the time the radius of the droplet develops to $r_{0}$, which is
definitely not $0$.

To calculate the initial radius, we have assumed that the nucleus is a perfect
sphere. Thus, the initial radius can calculated by
Equation~\eqref{eq:InitialRadius}.

\begin{equation} \label{eq:InitialRadius}
    \rho_{NaCl}\frac{4}{3}\pi r_{i}^{3}=m_{NaCl}
\end{equation}

We have used $\rho_{NaCl}=2160 \frac{kg}{m^{3}}$. This adjustment introduced an
effect of the nucleus mass and decreased the overall error of our
approximation.

The secondary improvement was on $e_{s}$. The approximation assumed $e_{s}$ to
be a constant, but normally it is not. Especially for small $r$ values,
\cite{Best} states that $e_{s}$ deviates from its constant value significantly.
We have applied two modifications on $e_{s}$. The first adjustment is the usage
of Raoult's law to incorporate the effect of surface tension, introducing an
additional factor depending on both $r$ and the mass of the nucleus. This
factor is shown in Equation~\eqref{eq:RaoultsLaw}. $i$ is a constant depending
on the molecular structure of the nucleus core and is taken as $i=2$.

\begin{equation}
    \label{eq:RaoultsLaw}
    \dfrac{e_{s}}{e_{s\infty}}=\dfrac{n_{H2O}}{i n_{solt}+n_{H2O}}=\dfrac{\dfrac{\dfrac{4}{3}\pi r^{3} \rho_{H2O}}{M_{H2O}}}{i \dfrac{m_{solt}}{M_{solt}} + \dfrac{\dfrac{4}{3}\pi r^{3} \rho_{H2O}}{M_{H2O}}}
\end{equation}

The second modification is the introduction of the effect of curvature, which
is shown in Equation~\eqref{eq:Curvature}.

\begin{equation}
    \label{eq:Curvature}
    \dfrac{e_{s}}{e_{s\infty}}=\exp \left( \dfrac{2\sigma_{lv}}{\rho_{l}R_{v}Tr} \right)
\end{equation}

The combined effect of surface tension and the curvature are given in
Equation~\eqref{eq:EsDefinition}.

\begin{equation}
    \label{eq:EsDefinition}
    e_{s}(r,T)=e_{s\infty}\left(\dfrac{\dfrac{\dfrac{4}{3}\pi r^{3} \rho_{H2O}}{M_{H2O}}}{i \dfrac{m_{solt}}{M_{solt}} + \dfrac{\dfrac{4}{3}\pi r^{3} \rho_{H2O}}{M_{H2O}}}\right) \exp \left( \dfrac{2\sigma_{lv}}{\rho_{l}R_{v}Tr} \right)
\end{equation}

A final modification has been applied on the numerator $(S-1)$ as recommended
by \cite{Mason}. The numerator is redefined in Equation~\eqref{eq:Numerator} to
add surface tension and curvature effects.

\begin{equation}
    \label{eq:Numerator}
    (S-1) \rightarrow (S-1)+\dfrac{2\sigma_{lv}}{\rho_{l}R_{v}Tr}-\dfrac{i n_{solt}}{i n_{solt}+n_{H2O}}
\end{equation}

Although this equation increases the accuracy of the results, it depends itself
on $r$. This means that the simple quadratic solution can not be used anymore
and we need a numerical integration method to solve the integral. The explicit
Euler method has been used for integration. The iterative steps are defined in
Equation~\eqref{eq:Iteration}.

\begin{equation}
    \label{eq:Iteration}
    r_{k+1}= \dfrac{S - 1+\dfrac{2\sigma_{lv}}{\rho_{l}R_{v}Tr}-\dfrac{n_{solt}}{i n_{solt}+n_{H2O}}}{K(T) + D(r_{k},T)}\dfrac{\Delta t}{r_{k}}
\end{equation}

The initial condition at $k=0$ is $r=r_{i}$ which was calculated as the radius
of the nucleus core itself.

Although this approximation is close to the values in
Table~\ref{tab:Table5.5}, it has its drawbacks with respect to reality. The
first limitation is that we have assumed the temperature to be constant
throughout the droplet and same as the environment. The second drawback is that
we have assumed that the environment properties like $\kappa, D_{v}, L_{lv}$
stay constant, which introduces an extra error into our approximation.

Another drawback is due to the numeric integration error. We have computed the
symbolic integral via Wolfram and compared its result with the numeric
integration result, but the difference was small enough to be negligible. An
insignificant approximation is on the shape of both the nucleus core and the
droplet, as these will deviate from spheres in real cases.


\subsection{Model Parameters}
$K$ is the thermodynamic term related to the latent heat release due to
condensation and diffusion of heat away from the droplet. It depends on the
ambient temperature $T$, latent heat of vaporization $L_{lv}$ and the thermal
conductivity coefficient $\kappa$. $\rho_l$ denotes the water density.
$\kappa$ is the coefficient of thermal conductivity in air and increases with
increasing temperature to account for the faster heat conduction at higher
temperatures due to the higher kinetic energy of the air molecules.

$D$ is the vapor diffusion term related to the diffusion of water vapor onto
the growing droplet. It depends on the saturation vapor pressure $e_{s}$,
temperature $T$, and the coefficient of water vapor diffusion in air $D_v$.
$D_v$ increases with increasing temperature but decreases with increasing
pressure. An increase in pressure means a decrease in the mean free path of the
molecules so that collisions with other molecules and particles occur more
frequently. The increase of  $D_v$ with increasing temperature again accounts
for the faster diffusion at higher temperatures.

Plus, $S$ is saturation ratio. So if $S$ is less than 1, $\frac{dr}{dt}$ is
less than 0, which describes the evaporation of a cloud drop. In the other way
when $S$ is greater than 1, it describes the condensation of a cloud drop.

Back to our equation of drop growth rate, $K$ depends on temperature and $D$
depends on temperature and pressure. The higher the temperature, the smaller is
the term $K$. $D$ has temperature dependent terms both in the nominator and in
the denominator.  Here the temperature dependence in the denominator dominates
so that $K$ also decreases with increasing temperature. This means that the
denominator is smaller for higher temperatures, i.e.\ the droplets grow or
shrink faster at higher temperatures for a given saturation ratio $S$. This is
intuitive because at higher temperature, the absolute difference in water vapor
pressure between the droplet surface and the surroundings is higher for a given
saturation ratio, and therefore also the water vapor gradient at the droplet
surface, which drives diffusional growth or evaporation of the droplet.

Diffusive transport in droplet growth process is smaller than advective
transport, but plays an important role in the process because  water vapor is
diffused towards the cloud drop, and hear is diffused away from the cloud drop.
In sum, the droplet growth rate is a function of saturation ratio over the sum
of rate of transfer of water vapor to the droplet and rate of transfer of
latent heat of condensation away from the droplet.


%--------------------------------------------------
% RESULTS
%--------------------------------------------------
\section{Results of Modeling Study}
Figure 1 shows a visual comparison between the droplet growth rate
modeled by Equation~\eqref{eq:5.27} with $r_0 = 0.75 \mu m$, $T=273 K$, $p=900
mb$, and $(S - 1) = 0.05\%$. A major assumption of Equation~\eqref{eq:5.27} is
that the curvature and solute effects are neglible once the droplet grows
beyond a few microns. The figure shows general agreement between the model and
values from Table 5.5 of \cite{Curry}.

To study the sensitivity of the model, we varied $T$ and $(S - 1)$, separately,
while holding all other parameters constant. Figure 2 and 3 show the results.
Increasing $T$ increases the droplet growth rate, due to decreased values of
$K$ and increased values of $D$.  Likewise, increasing $(S - 1)$ increases the
droplet growth rate due to an increased amount of water vapor, which lowers the
activation.

\begin{figure}
    \centering
    \includegraphics[width=\textwidth]{r_t.pdf}
    \label{fig:r_t}
    \caption{Droplet growth rate ignoring curvature and solute effects. The solid black line is the model, while the solid gray triangle, black outlined square, and solid gray circle are the radii from Table 5.5 of \cite{Curry} for $m_{solt}$ = 10$^{-12}$, 10$^{-13}$, and 10$^{-14}$ [g], respectively.}
\end{figure}


\begin{figure}
    \centering
    \includegraphics[width=\textwidth]{r_t_temperature.pdf}
    \label{fig:temperature}
    \caption{Droplet growth rate as the ambient temperature ($T$) is varied. The lines correspond to ambient temperatures of 273 K (solid black), 283 K (dotted black), 293 K (solid gray), and 303 K (dashed gray).}
\end{figure}


\begin{figure}
    \centering
    \includegraphics[width=\textwidth]{r_t_supersaturation.pdf}
    \label{fig:supersaturation}
    \caption{Droplet growth rate for multiple values of supersaturation ($S - 1$): 0.05\% (solid black), 0.10\% (dotted black), 0.15\% (solid gray), and 0.20\% (dashed gray).}
\end{figure}


%--------------------------------------------------
% CONCLUSION
%--------------------------------------------------
\section{Conclusion}
We have evaluated a simplified model of droplet growth rate in warm clouds.
Although the model ignores curvature and solute effects, the model is still
able to capture the general trends of droplet growth, as validated against
droplet growth values from \cite{Curry, Mason, Best} that include curvature and
solute effects. Sensitivity analysis of the model to changes in ambient temperature $T$
and supersaturation $(S -1)$ show a positive relationship between the droplet growth rate
and $T$, and the droplet growth rate and $(S - 1)$.



%--------------------------------------------------
% BIBLIOGRAPHY
%--------------------------------------------------
\bibliography{sources}
\bibliographystyle{plainnat}


\end{document}
