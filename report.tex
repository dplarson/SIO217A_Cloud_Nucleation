\documentclass[]{article}

\usepackage[margin=1.0in]{geometry}

% Equations
\usepackage{amsmath}

% Better tables
\usepackage{booktabs}
\usepackage{multirow}
\newcommand{\ra}[1]{\renewcommand{\arraystretch}{#1}}


\title{SIO 217A: Cloud Nucleation}
\author{Bengu Ozge Akyurek, David Larson, Guangchao Wang}
\date{\today}

\begin{document}

\maketitle

\section{Droplet Growth Rate}

The growth rate of a droplet by diffusion can be approximated as
\begin{align}
    \label{eq:5.26}
    r \frac{dr}{dt} = (S - 1) \left[ \frac{L_{lv}^2 \rho_l}{\kappa R_v T^2} + \frac{\rho_l R_v T}{e_s(T) D_v} \right] ^{-1} = \frac{S - 1}{K + D}
\end{align}
where $K$ and $D$ are the thermodynamic terms associated with heat conduction
and diffusion of water vapor, respectively.

Assuming the atmospheric ambient conditions are constant (i.e. $S$, $K$, and
$D$ are constant), then Equation~\ref{eq:5.26} can be integrated to get
\begin{align}
    \label{eq:5.27}
    r(t) = \left[ r_0^2 + \frac{2(S -1)}{K + D}(t - t_0)^2 \right] ,
\end{align}
which can then be rearranged to find $t$

\begin{align}
    t = (r^2 - r_0^2) \frac{K + D}{2(S - 1)}
\end{align}

\begin{table}[h]
    \centering
    \caption{Growth rate of droplets with nuclei of NaCl.}

    \ra{1.2}
    \begin{tabular}{@{} c r r r @{}}
        \\
        \toprule
        m [g] & $10^{-14}$ & $10^{-13}$ & $10^{-12}$ \\
        \midrule
        r [$\mu$m] & \multicolumn{3}{c}{t [s]} \\
        \midrule
        1  & 2.4    & 0.15   & 0.013 \\
        2  & 130    & 7.0    & 0.61 \\
        4  & 1,000  & 320    & 62 \\
        10 & 2,700  & 1,800  & 870 \\
        20 & 8,500  & 7,400  & 5,900 \\
        30 & 17,500 & 16,000 & 14,500 \\
        50 & 44,500 & 43,500 & 41,500 \\
        \bottomrule
    \end{tabular}
\end{table}

\end{document}
