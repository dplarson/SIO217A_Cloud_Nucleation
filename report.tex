\documentclass[]{article}

\usepackage[margin=1.0in]{geometry}

% Equations
\usepackage{amsmath}

% Better tables
\usepackage{booktabs}
\usepackage{multirow}
\newcommand{\ra}[1]{\renewcommand{\arraystretch}{#1}}


%--------------------------------------------------
% TITLE
%--------------------------------------------------
\title{SIO 217A: Cloud Droplet Growth}
\author{Team MecheE \\ Bengu Ozge Akyurek, David Larson, Guangchao Wang}
\date{\today}

\begin{document}

\maketitle


%--------------------------------------------------
% INTRO
%--------------------------------------------------
\section{Introduction}
Brief summary of how the topic affects atmospheric thermodynamics.


%--------------------------------------------------
% MODEL
%--------------------------------------------------
\section{Cloud Droplet Growth}

\subsection{Model Description}
Description of the model, its assumptions, and its possible shortcomings.


The growth rate of a droplet by diffusion can be approximated as
\begin{align}
    \label{eq:5.26}
    r \frac{dr}{dt} = (S - 1) \left[ \frac{L_{lv}^2 \rho_l}{\kappa R_v T^2} + \frac{\rho_l R_v T}{e_s(T) D_v} \right] ^{-1} = \frac{S - 1}{K + D}
\end{align}
where $K$ and $D$ are the thermodynamic terms associated with heat conduction
and diffusion of water vapor, respectively.

Assuming the atmospheric ambient conditions are constant (i.e. $S$, $K$, and
$D$ are constant), then Equation~\ref{eq:5.26} can be integrated to get
\begin{align}
    \label{eq:5.27}
    r(t) = \left[ r_0^2 + \frac{2(S -1)}{K + D}(t - t_0)^2 \right] ^{1/2},
\end{align}
which can then be rearranged to find $t$

\begin{align}
    t = (r^2 - r_0^2) \frac{K + D}{2(S - 1)}
\end{align}

Using the values from Table~\ref{tab:parameters}


\subsection{Model Parameters}
Physical significance of the model parameters you are calculating and varying.


%--------------------------------------------------
% RESULTS
%--------------------------------------------------
\section{Results of Modeling Study}
Conclusions you draw from your modeling study, with as much physical insight
as the modeling results allow.


%--------------------------------------------------
% CONCLUSION
%--------------------------------------------------
\section{Conclusion}


%--------------------------------------------------
% BIBLIOGRAPHY
%--------------------------------------------------
\bibliography{sources}
\bibliographystyle{plain}


%--------------------------------------------------
% APPENDIX
%--------------------------------------------------
\appendix

%-------------------------
\section{Figures}



%-------------------------
\section{Tables}

\begin{table}[h]
    \centering
    \caption{Paramters for droplet growth rate.}
    \label{tab:parameters}

    \begin{tabular}{l l l l}
    \toprule
    Parameter & Value & Units & Notes\\
    \midrule
    $S - 1$   & 0.05            & \%                         & \\
    $p$       & 900             & $mb$                       & \\
    $T$       & 273             & $K$                        & \\
    $r_0$     & 0.75            & $\mu m$                    & \\
    $L_{lv}$  & 2.5x10$^{6}$    & $J \ kg^{-1}$              & pure water at $T=273 K$ \\
    $\rho_l$  & 1000            & $kg \ m^{-3}$              & pure water at $T=273 K$ \\
    $R_v$     & 461             & $J \ kg^{-1} K^{-1}$       & \\
    $\kappa$  & 2.4x10$^{-2}$   & $J \ m^{-1} s^{-1} K^{-1}$ & $T=273 K$ \\
    $D_v$     & 2.21x10$^{-5}$  & $m^2 s^{-1}$               & $T=273 K$, $p=1000 mb$ \\
              & 2.46x10$^{-5}$  & $m^2 s^{-1}$               & $T=273 K$, $p=900 mb$ \\
    $e_s (T)$ & 6.15            & $mb$                       & $T=273 K$\\
    \bottomrule
    \end{tabular}
\end{table}


\begin{table}[h]
    \centering
    \caption{Growth rate of droplets with nuclei of NaCl, ($S - 1$)=0.05\%,
        $p$=900mb, $T$=273K, and $r_0$=0.75$\mu$m recreated from Table 5.5 of \cite{Curry}.}

    \ra{1.2}
    \begin{tabular}{@{} c r r r @{}}
        \\
        \toprule
        m [g] & $10^{-14}$ & $10^{-13}$ & $10^{-12}$ \\
        \midrule
        r [$\mu$m] & \multicolumn{3}{c}{t [s]} \\
        \midrule
        1  & 2.4    & 0.15   & 0.013 \\
        2  & 130    & 7.0    & 0.61 \\
        4  & 1,000  & 320    & 62 \\
        10 & 2,700  & 1,800  & 870 \\
        20 & 8,500  & 7,400  & 5,900 \\
        30 & 17,500 & 16,000 & 14,500 \\
        50 & 44,500 & 43,500 & 41,500 \\
        \bottomrule
    \end{tabular}
\end{table}

\end{document}
